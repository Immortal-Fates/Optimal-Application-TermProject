%! Mode:: "TeX:UTF-8"
%! TEX program = xelatex
\PassOptionsToPackage{quiet}{xeCJK}
\documentclass[withoutpreface,bwprint]{thesis-config}
\usepackage{third}
\usepackage{float}

\begin{document}
\maketitlepage{DDP 变长序列动态批调度}{杨逍宇3220105453}{\today}{何衍} %封面页 

% 摘要页
\newpage
\phantomsection
\addcontentsline{toc}{section}{摘要}

\begin{center}
{\zihao{3}\heiti 摘要}
\end{center}
\vspace{1em}

在同步式分布式数据并行(DDP)训练中,单步耗时由最慢 GPU 决定。NLP 任务的变长序列与动态 padding 使各卡有效 token 数与显存峰值产生显著波动,形成“慢卡”瓶颈并拉长尾延迟。本文将“每步 per-GPU micro-batch 分配”形式化为小规模 min-max 整数优化问题,提出一种动态调度策略:在每步开始前预取候选样本,构造基于 padded tokens 的成本代理,通过枚举求解最优 per-rank batch size,并结合 DDP 梯度缩放保证全局平均梯度一致性。

在 4 卡 DDP + SST-2 + DistilBERT 的快速实验中,动态调度显著降低了跨卡 padded tokens 的方差,并在较高 global batch 规模下实现 step time 的 P95 改善与吞吐提升。该方法不依赖复杂外部框架,工程实现轻量,能较好体现“数学规划在工程实践中的落地价值”。

\vspace{1em}
\noindent\textbf{关键词:}分布式训练\quad 变长序列\quad min-max 整数优化\quad 动态调度\quad PyTorch DDP

% 目录页
\newpage
\phantomsection
\pdfbookmark[0]{目录}{toc}
\tableofcontents
\newpage

\section{前言}

\subsection{问题背景}

在 DDP 同步训练中,所有 GPU 需要在每一步结束时完成梯度同步,因此整体 step time 由最慢的 rank 决定\cite{PyTorchDDP}。对于 NLP 任务,样本长度分布具有明显长尾特征;动态 padding 使不同 rank 的有效 token 数不一致,从而导致计算量、显存占用与通信等待呈现随机波动。固定 per-rank batch size 虽易实现,但在高并发训练下会放大“慢卡效应”,影响吞吐与稳定性。另一方面,Transformer 的注意力计算复杂度与序列长度呈平方相关\cite{Vaswani2017},使得长度波动在计算代价上被进一步放大。

\subsection{研究意义}

该问题具有直接的工程价值:在保持模型精度不变的前提下,通过调度策略降低尾延迟、提升 GPU 利用率,能显著提高分布式训练的效率。更重要的是,该问题天然适合数学规划建模:每步决策变量为各卡样本数,目标为最小化最大开销,约束为全局 batch 与显存上限。这一过程完整呈现了“问题抽象—模型建立—算法求解—工程验证”的优化闭环,契合课程对数学规划应用与计算求解能力的要求。

\subsection{相关基础知识与公开资料}

本文参考并总结了官方文档与公开资料中的基础知识:
\begin{itemize}
\item \textbf{DDP 同步机制}:梯度 allreduce 的平均化会在不同 local batch 下产生偏差,需要额外的 loss 缩放以保持全局平均梯度一致\cite{PyTorchDDP}。
\item \textbf{变长序列与 padding}:序列长度决定 attention 计算代价与显存占用,padded tokens 可作为合理的代理成本\cite{Vaswani2017}。
\item \textbf{数据与模型}:采用 GLUE/SST-2 数据集与 DistilBERT 模型,相关使用方式可见 Hugging Face 的 Datasets 与 Transformers 文档\cite{HFDatasets,HFTransformers,GLUEBenchmark,DistilBERT}。
\end{itemize}

这些公开资料为实验搭建提供了可复现的工程基础,同时也为问题建模提供了清晰的物理含义。

\subsection{研究内容与论文结构}

本文的主要研究内容如下:
\begin{enumerate}
\item 将 per-step micro-batch 分配形式化为 min-max 整数规划,并引入可计算的成本代理;
\item 在 $G=4$ 的场景下采用 exact 枚举求解,并加入方差作为 tie-break;
\item 结合 DDP 梯度缩放推导,保证动态 batch 下梯度等价;
\item 在 SST-2 上进行静态与动态对比实验,评估 step time、吞吐与跨卡均衡性;
\item 总结失败设置与实践经验,为后续改进提供方向。
\end{enumerate}

论文结构安排如下:第二章建立数学模型;第三章给出求解方法与实现细节;第四章进行案例分析;第五章总结结论与亮点并讨论局限与展望。

\section{问题建模}

\subsection{建模假设与符号定义}

设总 GPU 数为 $G=4$,第 $t$ 步第 $g$ 张 GPU 的 micro-batch 大小为 $b_{g,t}$,全局目标 batch 为 $B$。设每卡最大可承受 batch 上限为 $b_g^{\max}$(由 OOM 探测得到)。DDP 同步训练的关键在于最慢 GPU 的开销决定整步耗时,因此自然对应于 min-max 目标。

\subsection{成本代理的构造与合理性}

每个 rank 在本步开始前预取 $K$ 个样本,记前 $k$ 个样本的最大长度为 $\max\_len[k]$,定义
\[
\text{padded\_tokens}[k] = k \cdot \max\_len[k].
\]
该量与注意力计算复杂度一致性较好\cite{Vaswani2017},因此可作为计算与显存的统一代理。本文采用两种常见代理形式:
\[
 c_g(k) = \text{padded\_tokens}[k],\quad \text{或}\quad c_g(k) = (\text{padded\_tokens}[k])^2.
\]
前者更稳健、后者更强调长序列惩罚。该代理单调可计算,适合在 step 前快速估算。
\subsection{min-max 整数优化模型}

对每一步,求解以下问题:
\[
\min_{b_1,\dots,b_G} \max_g c_g(b_g)
\quad \text{s.t.} \sum_g b_g = B,
\ 1 \le b_g \le b_g^{\max},\ b_g \in \mathbb{Z}.
\]
当多个解具有相同最大成本时,进一步选择成本方差 $\mathrm{Var}(c_g(b_g))$ 更小的方案,以增强均衡性。该模型可视为对“最慢 GPU 瓶颈”的直接优化,并与 DDP 同步机制相一致。

\subsection{与 DDP 梯度等价性的关系}

当各 rank 的 local batch 不同,若直接 allreduce 平均梯度会改变有效学习率。设每个 rank 使用 mean loss $L_g$,全局 batch 为 $B=\sum_g b_g$,则应在 backward 前缩放:
\[
\tilde L_g = L_g \cdot \frac{G \cdot b_g}{B}.
\]
此时 allreduce 平均后的梯度与全局平均梯度等价,确保优化目标不受动态 batch 分配影响。

\section{求解方法}

\subsection{预取 buffer 与 exact 枚举}

由于 $G=4$ 且 $B$ 不大,本文采用 exact 枚举求解。具体流程:
\begin{enumerate}
\item 每个 rank 预取 $K$ 个样本形成 buffer,计算 $c_g(k)$ 序列;
\item 使用 all-gather 将成本数组汇总到 rank0;
\item rank0 枚举所有满足 $\sum b_g = B$ 的组合,选择 min-max + 最小方差解;
\item 广播 $b_g$ 给各 rank,按分配取样训练。
\end{enumerate}

\subsection{算法复杂度与可实现性}

在 $G=4$ 的场景下,枚举规模随 $B$ 的增长近似为 $O(B^{G-1})$,对本文设定的 batch 范围而言完全可接受。相比启发式调度,exact 枚举具有确定性、可解释性强的优势,适合作为课程项目展示优化求解亮点。

\subsection{工程实现与工具链}

系统实现基于 PyTorch DDP 与 Hugging Face Transformers/Datasets\cite{HFTransformers,HFDatasets},数据集为 GLUE/SST-2\cite{GLUEBenchmark},模型为 DistilBERT\cite{DistilBERT}。训练过程记录 per-step 指标并输出到 CSV,分析脚本基于 pandas 与 matplotlib 自动生成图表与汇总表。代码结构遵循模块化设计,调度算法与训练逻辑解耦,便于复现与扩展。

\subsection{关键工程要点}

\begin{itemize}
\item \textbf{OOM 保护}:通过 max\_length 截断与启动时 batch 探测得到 $b_g^{\max}$。
\item \textbf{动态 loss 缩放}:按 $\tilde L_g = L_g \cdot G b_g / B$ 缩放 loss,保持全局平均梯度等价。
\item \textbf{指标记录}:每步记录 step time、peak memory、padded tokens、吞吐等,便于后续统计分析。
\end{itemize}

\section{案例分析}

\subsection{硬件与软件环境}

实验在单机 4 卡环境上完成,GPU 为 4 张 NVIDIA RTX 3090。图\ref{fig:nsmi1} 和图\ref{fig:nsmi2} 分别展示优化前后的 nvidia-smi 截图,用于对比 GPU 利用率(GPU-util)与显存占用的变化,从而直观体现调度策略在资源均衡与利用率方面的优势。

\begin{figure}[H]
\centering
\includegraphics[width=0.9\textwidth]{figures/nvidia-smi1.png}
\caption{优化前 nvidia-smi 截图(GPU-util 对比)}
\label{fig:nsmi1}
\end{figure}

\begin{figure}[H]
\centering
\includegraphics[width=0.9\textwidth]{figures/nvidia-smi2.png}
\caption{优化后 nvidia-smi 截图(GPU-util 对比)}
\label{fig:nsmi2}
\end{figure}

软件环境基于 Python + PyTorch DDP 与 Hugging Face 生态,训练与分析脚本均可在同一环境下运行。

\subsection{实验设置}

\begin{itemize}
\item 数据集:SST-2(GLUE)
\item 模型:distilbert-base-uncased
\item 训练配置:4 GPU,global batch $B=64$,steps=800,max\_length=256
\item Baseline:static(每卡固定 16)
\item 方法:dynamic(per-step min-max 调度,cost=\texttt{tokens})
\item buffer size:64,支持更稳定的成本估计
\end{itemize}

\subsection{指标体系}

本文关注三类指标:
\begin{enumerate}
\item \textbf{时延类}:每步最大 step time 的均值与 P95;
\item \textbf{吞吐类}:samples/s 与 tokens/s 的均值与 P95;
\item \textbf{均衡性}:跨卡的 step time、padded tokens 与 peak memory 标准差。
\end{enumerate}

\subsection{结果与讨论}

表\ref{tab:summary} 汇总了静态与动态调度的对比结果。动态调度在时延和吞吐上均有改善:mean max step time 降低 3.70\%,P95 降低 3.82\%;samples/s 平均提升 3.85\%,P95 提升 4.16\%。更显著的是,跨卡 padded tokens 标准差降低 71.60\%,表明动态调度有效缓解了长度不均衡。需要注意的是,peak memory 标准差略有上升(约 4.12\%),说明在某些步长分配更激进时,仍可能引入显存波动,这为后续调度策略改进提供了方向。

\begin{table}[H]
\centering
\caption{Static 与 Dynamic 对比结果}
\label{tab:summary}
\begin{tabular}{lccc}
\hline
指标 & Static & Dynamic & 提升(\%) \\
\hline
Mean max step time (ms) & 156.80 & 150.99 & -3.70 \\
P95 max step time (ms) & 161.37 & 155.21 & -3.82 \\
Mean throughput (samples/s) & 408.27 & 423.97 & +3.85 \\
P95 throughput (samples/s) & 418.67 & 436.10 & +4.16 \\
Mean throughput (tokens/s) & 14261.50 & 14456.85 & +1.37 \\
P95 throughput (tokens/s) & 17276.87 & 17966.80 & +3.99 \\
Mean std(step time) & 0.327 & 0.307 & -6.10 \\
Mean std(padded tokens) & 112.06 & 31.83 & -71.60 \\
Mean std(peak mem) & 1.422 & 1.481 & +4.12 \\
\hline
\end{tabular}
\end{table}

\subsection{图表分析}

图\ref{fig:maxstep} 展示了每步最大 step time 的时间序列,动态调度在大多数步骤上保持更低的波动;图\ref{fig:stdpad} 展示了 padded tokens 的跨卡标准差,动态方法在全程显著低于 static。图\ref{fig:stdstep} 与图\ref{fig:stdmem} 分别展示 step time 与 peak memory 的跨卡标准差,体现了动态调度在时延均衡性上的整体改善。图\ref{fig:dist} 展示了 step time 分布,动态方法在尾部分布上更集中。

\begin{figure}[H]
\centering
\includegraphics[width=0.8\textwidth]{figures/max_step_time.png}
\caption{每步最大 step time(static vs dynamic)}
\label{fig:maxstep}
\end{figure}

\begin{figure}[H]
\centering
\includegraphics[width=0.8\textwidth]{figures/std_padded_tokens.png}
\caption{跨卡 padded tokens 标准差}
\label{fig:stdpad}
\end{figure}

\begin{figure}[H]
\centering
\includegraphics[width=0.8\textwidth]{figures/std_step_time.png}
\caption{跨卡 step time 标准差}
\label{fig:stdstep}
\end{figure}

\begin{figure}[H]
\centering
\includegraphics[width=0.8\textwidth]{figures/std_peak_mem_mb.png}
\caption{跨卡 peak memory 标准差}
\label{fig:stdmem}
\end{figure}

\begin{figure}[H]
\centering
\includegraphics[width=0.8\textwidth]{figures/step_time_distribution.png}
\caption{step time 分布(static vs dynamic)}
\label{fig:dist}
\end{figure}

\subsection{失败设置与经验总结}

在更激进的设置(如 cost=\texttt{tokens\_sq})下,虽然 padded tokens 的均衡性更好,但 step time 反而变差,说明代理成本与真实耗时之间仍存在偏差;调度策略需在“均衡性”与“调度开销”之间权衡。此外,过小的 buffer size 会降低成本估计精度,导致分配波动。该部分的负结果为后续改进提供了实践依据。

\section{结论与亮点}

本文针对 DDP 变长序列训练的负载不均衡问题,完成了“建模—求解—实验验证”的闭环。实验结果表明,动态 per-GPU micro-batch 调度在保持训练稳定的同时显著改善了跨卡均衡性,并提升了 step time 与吞吐表现。

\textbf{亮点与心得总结:}
\begin{enumerate}
\item 以真实工程问题为载体,完整呈现数学规划建模与求解过程;
\item 给出 min-max 整数优化的 exact 枚举解,并引入方差 tie-break 强化均衡性;
\item 推导并实现 DDP 梯度缩放,使动态 batch 与全局平均梯度等价;
\item 记录失败配置,分析代理成本与真实耗时的偏差来源;
\item 总结 PyTorch DDP、Transformers、数据缓存与离线运行的工具使用经验。
\end{enumerate}

\subsection*{局限与展望}

当前工作仍存在局限:代理成本未显式考虑通信与 kernel 融合;枚举方法仅适用于小规模 GPU;模型与数据规模较小。未来可考虑:引入更精细的成本模型,研究近似优化或启发式调度,并拓展到更大规模训练。

%%%%%%%%%%%%%%%%%%%%%%%%%%%%%%%%%%%%%%%%%%%%%%%%%%%%%%%%%%%%%
%% 参考文献
\clearpage
\bibliographystyle{gbt7714-numerical}
\bibliography{ref.bib}

\end{document}
